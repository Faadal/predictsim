\documentclass{article}
% The file ijcai11.sty is the style file for IJCAI-11 (same as ijcai07.sty).
\usepackage{ijcai11}

% Use the postscript times font!
\usepackage{times}
\usepackage{datetime}

\usepackage{mathptmx}
\usepackage[T1]{fontenc}

% the following package is optional:
%\usepackage{latexsym}

% Following comment is from ijcai97-submit.tex:
% The preparation of these files was supported by Schlumberger Palo Alto
% Research, AT\&T Bell Laboratories, and Morgan Kaufmann Publishers.
% Shirley Jowell, of Morgan Kaufmann Publishers, and Peter F.
% Patel-Schneider, of AT\&T Bell Laboratories collaborated on their
% preparation.

% These instructions can be modified and used in other conferences as long
% as credit to the authors and supporting agencies is retained, this notice
% is not changed, and further modification or reuse is not restricted.
% Neither Shirley Jowell nor Peter F. Patel-Schneider can be listed as
% contacts for providing assistance without their prior permission.

% To use for other conferences, change references to files and the
% conference appropriate and use other authors, contacts, publishers, and
% organizations.
% Also change the deadline and address for returning papers and the length and
% page charge instructions.
% Put where the files are available in the appropriate places.


\title{Master 1 MoSIG Research Project Report\\
Learning Job Runtimes in Homogenous Clusters}
\author{Valentin Reis\\
Supervised by: Denis Trystram \& Eric Gaussier.\\
}

\begin{document}

\maketitle

{% Mosig student
  {\hbox to0pt{\vbox{\baselineskip=10dd\hrule\hbox
to\hsize{\vrule\kern3pt\vbox{\kern3pt
\hbox{{\small I understand what plagiarism entails and I declare that this report }}
\hbox{{\small is my own, original work. }}
\hbox{{\small Name, date and signature: }}
\kern3pt
}\hfil%\kern3pt
\vrule
}\hrule}
}}
}


\begin{abstract}
  Lorem ipsum dolor sit amet, consetetur sadipscing elitr, sed diam nonumy eirmod
  tempor invidunt ut labore et dolore magna aliquyam erat, sed diam voluptua. At
  vero eos et accusam et justo duo dolores et ea rebum. Stet clita kasd gubergren,
  no sea takimata sanctus est Lorem ipsum dolor sit amet.
\end{abstract}

\section{Introduction}
High Performance Computing (HPC) systems are complex machinery at the frontier between research in scheduling and system administration.

The ephemeral nature and broad range of existing architecture of such systems make the development and application of theoretical results difficult.
For instance, the topological models of these systems regularly change on a yearly (even weekly, in the case of computing grids) basis, and the task of finding scheduling strategies which are robust to those changes is a current research problem.

In addition, the input data of the decisional part of these systems presents many peculiarities. For instance, and this is the focus of this research project, the run time of a given job on a specific system is seldom known in advance.

As a consequence of these difficulties most free, open-source and commercial resource management software use simple heuristics, which at best provide bounds on their performance, and at worst guarantee a few functional properties.
An example of such a strategy is the First Come First Serve (FCFS) strategy to schedule parallel jobs on a homogenous cluster of machines. Among other properties (such as robustness to weak information about job run times), this strategy guarantees the avoidance of starvation.

As hinted previously, one of the frontiers to apply more sophisticated techniques in order to schedule jobs on these systems is the uncertainty in the data provided by the users of those systems.
Most resource management software (including the SLURM, OpenPBS, OpenLava and OAR systems) do ask information about jobs to users, such as topological requests in terms of processing units and memory, the name of the executable, miscellaneous functional requirements and, last but not least, the expected run time (the user-provided estimate will be called \textbf{walltime} in the rest of the paper) of the job.
The the true run time ( now reffered as \textbf{runtime} ) of a job with respect to a given affected topology  is of great interest, as the scheduling policies are highly dependent on this data to provide good solutions.


\subsection{Problem Statement}
The general problem statement we are dealing with is the following:


\subsection{Word Processing Software}

\section{Motivation}

\subsection{Layout}


\subsection{Format of Electronic Manuscript}


\subsubsection{Blind Review}

In order to make blind reviewing possible, authors must omit their
names and affiliations when submitting the paper for review. In place
of names and affiliations, provide a list of content areas. When
referring to one's own work, use the third person rather than the
first person. For example, say, ``Previously,
Gottlob~\shortcite{gottlob:nonmon} has shown that\ldots'', rather
than, ``In our previous work~\cite{gottlob:nonmon}, we have shown
that\ldots'' Try to avoid including any information in the body of the
paper or references that would identify the authors or their
institutions. Such information can be added to the final camera-ready
version for publication.

\subsection{Abstract}

Place the abstract at the beginning of the first column 3$''$ from the
top of the page, unless that does not leave enough room for the title
and author information. Use a slightly smaller width than in the body
of the paper. Head the abstract with ``Abstract'' centered above the
body of the abstract in a 12-point bold font. The body of the abstract
should be in the same font as the body of the paper.

The abstract should be a concise, one-paragraph summary describing the
general thesis and conclusion of your paper. A reader should be able
to learn the purpose of the paper and the reason for its importance
from the abstract. The abstract should be no more than 200 words long.

\subsection{Text}

The main body of the text immediately follows the abstract. Use
10-point type in a clear, readable font with 1-point leading (10 on
11).

Indent when starting a new paragraph, except after major headings.

\subsection{Headings and Sections}

When necessary, headings should be used to separate major sections of
your paper. (These instructions use many headings to demonstrate their
appearance; your paper should have fewer headings.)

\subsubsection{Section Headings}

Print section headings in 12-point bold type in the style shown in
these instructions. Leave a blank space of approximately 10 points
above and 4 points below section headings.  Number sections with
arabic numerals.

\subsubsection{Subsection Headings}

Print subsection headings in 11-point bold type. Leave a blank space
of approximately 8 points above and 3 points below subsection
headings. Number subsections with the section number and the
subsection number (in arabic numerals) separated by a
period.

\subsubsection{Subsubsection Headings}

Print subsubsection headings in 10-point bold type. Leave a blank
space of approximately 6 points above subsubsection headings. Do not
number subsubsections.

\subsubsection{Special Sections}

You may include an unnumbered acknowledgments section, including
acknowledgments of help from colleagues, financial support, and
permission to publish.

Any appendices directly follow the text and look like sections, except
that they are numbered with capital letters instead of arabic
numerals.

The references section is headed ``References,'' printed in the same
style as a section heading but without a number. A sample list of
references is given at the end of these instructions. Use a consistent
format for references, such as that provided by Bib\TeX{}. The reference
list should not include unpublished work.

\subsection{Citations}

Citations within the text should include the author's last name and
the year of publication, for example~\cite{gottlob:nonmon}.  Append
lowercase letters to the year in cases of ambiguity.  Treat multiple
authors as in the following examples:~\cite{abelson-et-al:scheme}
or~\cite{bgf:Lixto} (for more than two authors) and
\cite{brachman-schmolze:kl-one} (for two authors).  If the author
portion of a citation is obvious, omit it, e.g.,
Nebel~\shortcite{nebel:jair-2000}.  Collapse multiple citations as
follows:~\cite{gls:hypertrees,levesque:functional-foundations}.
\nocite{abelson-et-al:scheme}
\nocite{bgf:Lixto}
\nocite{brachman-schmolze:kl-one}
\nocite{gottlob:nonmon}
\nocite{gls:hypertrees}
\nocite{levesque:functional-foundations}
\nocite{levesque:belief}
\nocite{nebel:jair-2000}

\subsection{Footnotes}

Place footnotes at the bottom of the page in a 9-point font.  Refer to
them with superscript numbers.\footnote{This is how your footnotes
should appear.} Separate them from the text by a short
line.\footnote{Note the line separating these footnotes from the
text.} Avoid footnotes as much as possible; they interrupt the flow of
the text.

\section{Illustrations}

Place all illustrations (figures, drawings, tables, and photographs)
throughout the paper at the places where they are first discussed,
rather than at the end of the paper. If placed at the bottom or top of
a page, illustrations may run across both columns.

Illustrations must be rendered electronically or scanned and placed
directly in your document. All illustrations should be in black and
white, as color illustrations may cause problems. Line weights should
be 1/2-point or thicker. Avoid screens and superimposing type on
patterns as these effects may not reproduce well.

Number illustrations sequentially. Use references of the following
form: Figure 1, Table 2, etc. Place illustration numbers and captions
under illustrations. Leave a margin of 1/4-inch around the area
covered by the illustration and caption.  Use 9-point type for
captions, labels, and other text in illustrations.

\section*{Acknowledgments}

The preparation of these instructions and the \LaTeX{} and Bib\TeX{}
files that implement them was supported by Schlumberger Palo Alto
Research, AT\&T Bell Laboratories, and Morgan Kaufmann Publishers.
Preparation of the Microsoft Word file was supported by IJCAI.  An
early version of this document was created by Shirley Jowell and Peter
F. Patel-Schneider.  It was subsequently modified by Jennifer
Ballentine and Thomas Dean, Bernhard Nebel, and Daniel Pagenstecher.
These instructions are the same as the ones for IJCAI--05, prepared by
Kurt Steinkraus, Massachusetts Institute of Technology, Computer
Science and Artificial Intelligence Lab.

\appendix

\section{\LaTeX{} and Word Style Files}\label{stylefiles}

The \LaTeX{} and Word style files are available on the IJCAI--11
website, {\tt http://www.ijcai-11.org/}.
These style files implement the formatting instructions in this
document.

The \LaTeX{} files are {\tt ijcai11.sty} and {\tt ijcai11.tex}, and
the Bib\TeX{} files are {\tt named.bst} and {\tt ijcai11.bib}. The
\LaTeX{} style file is for version 2e of \LaTeX{}, and the Bib\TeX{}
style file is for version 0.99c of Bib\TeX{} ({\em not} version
0.98i). The {\tt ijcai11.sty} file is the same as the {\tt
ijcai07.sty} file used for IJCAI--07.

The Microsoft Word style file consists of a single file, {\tt
ijcai11.doc}. This template is the same as the one used for
IJCAI--07.

These Microsoft Word and \LaTeX{} files contain the source of the
present document and may serve as a formatting sample.

Further information on using these styles for the preparation of
papers for IJCAI--11 can be obtained by contacting {\tt
pcchair11@ijcai.org}.

%% The file named.bst is a bibliography style file for BibTeX 0.99c
\bibliographystyle{named}
\bibliography{ijcai11}

\end{document}

