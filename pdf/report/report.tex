\documentclass{article}
% The file ijcai11.sty is the style file for IJCAI-11 (same as ijcai07.sty).
\usepackage{ijcai11}

% Use the postscript times font!
\usepackage{times}
\usepackage{datetime}

% the following package is optional:
%\usepackage{latexsym}

% Following comment is from ijcai97-submit.tex:
% The preparation of these files was supported by Schlumberger Palo Alto
% Research, AT\&T Bell Laboratories, and Morgan Kaufmann Publishers.
% Shirley Jowell, of Morgan Kaufmann Publishers, and Peter F.
% Patel-Schneider, of AT\&T Bell Laboratories collaborated on their
% preparation.

% These instructions can be modified and used in other conferences as long
% as credit to the authors and supporting agencies is retained, this notice
% is not changed, and further modification or reuse is not restricted.
% Neither Shirley Jowell nor Peter F. Patel-Schneider can be listed as
% contacts for providing assistance without their prior permission.

% To use for other conferences, change references to files and the
% conference appropriate and use other authors, contacts, publishers, and
% organizations.
% Also change the deadline and address for returning papers and the length and
% page charge instructions.
% Put where the files are available in the appropriate places.


\title{Master 1 MoSIG Research Project Report\\
Learning Job Runtimes in Homogenous HPC Systems}
\author{Valentin Reis\\
Supervised by: Denis Trystram \& Eric Gaussier.\\
}

\begin{document}

\maketitle

{% Mosig student
  {\hbox to0pt{\vbox{\baselineskip=10dd\hrule\hbox
to\hsize{\vrule\kern3pt\vbox{\kern3pt
\hbox{{\small I understand what plagiarism entails and I declare that this report }}
\hbox{{\small is my own, original work. }}
\hbox{{\small Name, date and signature: }}
\kern3pt
}\hfil%\kern3pt
\vrule
}\hrule}
}}
}


\begin{abstract}
  Lorem ipsum dolor sit amet, consetetur sadipscing elitr, sed diam nonumy eirmod
  tempor invidunt ut labore et dolore magna aliquyam erat, sed diam voluptua. At
  vero eos et accusam et justo duo dolores et ea rebum. Stet clita kasd gubergren,
  no sea takimata sanctus est Lorem ipsum dolor sit amet.
\end{abstract}

\section{Introduction}
High Performance Computing (HPC) systems are complex machinery at the frontier between research in scheduling and systems engineering.

The ephemeral nature, and broad range of existing architectures of such systems make the development and application of theoretical results difficult.
New schemes for distributing resources (e.g.\ memory caches, hard drives, processing units, RAM memory) are ubiquitous. Moreover, the topology of a HPC system can change on a monthly or weekly basis, with the addition of new hardware.
Finding scheduling, and resource management strategies which can adapt to those changes is a current research problem.

In addition, the input data these systems have to work with presents many peculiarities. The nature of the information which users provide is very often loose, ( e.g.\ with only upper and/or lower bounds provided ). For instance, and this will be the focus of this paper, the run time of a given job on a specific system is seldom known in advance, however an upper bound may be given by the user.

As a consequence of these difficulties most free, open-source and commercial resource management software use simple heuristics, which at best provide bounds on their performance, and at worst guarantee a few functional properties.
An example of such a strategy is the First Come First Serve (FCFS) strategy to schedule parallel jobs on a homogenous cluster of machines. Among other properties (such as robustness to weak information about the amount of time a job will run on the system), this strategy guarantees the avoidance of starvation.

As mentionned previously, one of the aspects that must be dealt with in order to apply more sophisticated techniques to schedule jobs on these systems is the uncertainty in the data provided by the users of those systems.

\subsection{Research Direction}
The general direction we are headed in, in the course of this research project, is to deal with the input data of the resource management systems.
In particular, we seek to apply Machine Learning techniques in order to reduce uncertainty in the data and extract information and/or structure relevant for subsequent use by a scheduling algorithm. We will be working only with the problem of presenting input data in the most valuable way possible to a scheduling algorithm, and will not deal with how to use this data beyond simple cases. As for showing the added value of the specific the information we choose to extract, references from the HPC scheduling litterature will provide ground to stand on.

\subsection{Job runtimes}
Most HPC resource management software (including the SLURM, OpenPBS, OpenLava and OAR software) do ask information about jobs to users, such as topological requests in terms of processing units and memory, the name of the executable, miscellaneous functional requirements and, last but not least, the expected run time of the job. This user-provided estimate of the run time of a job on a specific system will be referred as \textbf{reqtime} in the rest of the paper. Most of these software use the \textbf{reqtime} of a job as an upper bound on its run time, and kill it should \textbf{reqtime} be violated. As a consequence, this information is over-estimated by the users, if they choose to provide it. The following section will look in depth at this relationship.

The true run time of a job with respect to a given affected topology is of great interest, as the scheduling policies are highly dependent on this information to provide good solutions \ref{truc}. We will refer to this quantity as the \textbf{runtime} of a job.
It must be clear that the \textbf{runtime} of a job is only defined with respect to a specific processing environment to which it might be affected.
This can include and is not limited to, the network topology of the processing units, the availability of shared memory, message passing costs, and the operating system supporting the computations.


\subsection{Problem Statement}
The general problem statement we are dealing with is the following:

TODO: predict runtime on a homogenous grid, blabla.

TODO: insofar as runtime is important, lets do this..

\section{Motivation}

\subsection{Importance of \textbf{runtime}}
\label{sub:importance_of_runtime}
TODO: to emphasise again... provide more references.

TODO: moreover, show existing solutions such as SLURM which use the walltime instead (leading to the next subsection)

% subsection importance_of_runtime (end)

\subsection{\textbf{reqtime} vs \textbf{runtime} on a real system}
\label{sub:reqtime_vs_runtime_on_a_real_system}

TODO: show the curie log.

% subsection reqtime_vs_runtime_on_a_real_system (end)


\section{State of the art in predicting \textbf{runtime}}


\subsection{Nature of the prediction}
\label{sub:nature_of_the_prediction}

TODO: -what to predict: value?, confidence interval?, distribution? whith which algorithms can we use those?

% subsection nature_of_the_prediction (end)

\subsection{Predicting a value}
\label{sub:predicting_a_value}

TODO: -give the references and explain the historical methods, gibbons historical scheduler, the tsafir et al paper with mean of two last runtimes values userwise..

% subsection predicting_a_value (end)

\subsection{Predicting a distribution}
\label{sub:predicting_a_distribution}

TODO: -give the references and explain the probabilistic backfilling thing..

% subsection predicting_a_distribution (end)


\section{Our approach}
\label{sec:our_approach}

\subsection{Random Forests}
\label{sub:random_forests}
TODO: -explain our approach, why it could lead to better results (external info+signal locality(ref hmm thesis for locality..))
% subsection random_forests (end)

\subsection{Explainability}
\label{sub:explainability}
TODO: -explain one advantage of random forests: discussion about the trees after learning.. 
% subsection explainability (end)

% section our_approach (end)

\section{Preliminary Results}
\label{sec:preliminary_results}

TODO:results..

% section preliminary_results (end)

\section{Conclusions}
\label{sec:conclusions}

TODO:conclure..

% section conclusions (end)


\section{Acknowledgements}
\label{sec:conclusions}

TODO:remercier..

% section conclusions (end)


%% The file named.bst is a bibliography style file for BibTeX 0.99c
\bibliographystyle{named}
\bibliography{report}

\end{document}

